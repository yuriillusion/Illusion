\documentclass{report}
\usepackage[left=20mm, top=15mm, right=20mm, bottom=15mm, nohead, nofoot]{geometry}
\usepackage[T2A]{fontenc}
\usepackage[utf8]{inputenc}
\usepackage[russian]{babel}
\usepackage{amsmath}
\usepackage{graphicx}
\graphicspath{{../}}
\DeclareGraphicsExtensions{.png}

\title{Вопросы по курсовой работе}
\author{Шклярик Ю.Н., 6 группа, 3 курс}

\begin{document}

\maketitle

\section{Задача для бесконечной проводящей нити}
Для рассчёта поля, порождаемого бесконечной проводящей нитью, я пришёл к следующему подходу: вместо уравнения Лапласа решать уравнение Пуассона $\Delta u=q\delta(x-x_0)\delta(y-y_0)$.

\section{Задача для бесконечной проводящей полоски}
Я придумал 2 подхода для рассчёта поля, порождаемого бесконечной полоской ширины $l$ с заданным потенциалом $U$:\\
1) Решать задачу на круглой области достаточно большого радиуса с вырезанным достаточно малым по ширине прямоугольником. В этом случае на границе прямоугольника задаётся граничное условие 1-ого рода со значением потенциала $U$, а на границе круга граничное условие задаётся с учётом того, что на бесконечности можно пренебречь размерами полоски, посчитав её бесконечной нитью.\\
2) Решать задачу на круглой области достаточно большого радиуса, где граничные условия на круге задаются аналогично предыдущему случаю, а в вершинах области, соответствующих полоске, устанавливается потенциал $U$, что вызывает некоторые трудности с построением сетки и заданием внутренних значений в области на отрезке, соответствующем полоске.\\
Проблема состоит в том, что я не понимаю, как выполнить переход от полоски к нити. То есть, какое поле порождала бы полоска с заданным потенциалом, если можно было бы пренебречь её размерами?\\
Также я не понимаю, зачем в этом случае использовать переход к цилиндрическим координатам? Ведь в этом случае, из-за отсутствия симметрии в цилиндрических координатах, двумерная задача останется двумерной, изменится только вид уравнения и область. Последнее, насколько я понимаю, вообще говоря, не обеспечивает выигрыш в объёме вычислений.

\section{Базовые принципы электростатики}
Столкнулся с непонимаением того, как распределяется заряд и потенциал (относительно бесконечности) в проводниках в отсутствии и присутствии внешнего электростатического поля. Хочется понять это на конкретных примерах.\\
1) Потенциал в точке расположения точечного заряда бесконечен? (следует из закона Кулона)\\
2) Потенциал бесконечной заряжённой нити расходится в точках расположения нити и на бесконечности? (т.к. фундаментальное решение $\ln{\cfrac{1}{r}}$)\\
3) Бесконечная заряжённая пластина имеет потенциал $U$, внешнее поле отсутствует. Как может быть распределён заряд по поверхности?\\
4) Тот же случай для заряженного шара.\\
5) В объёмных проводниках заряд всегда распределяется по поверхности или только в присутствии электростатического поля?\\
6) В электростатике все проводники имеют константный потенциал, иначе была бы разность потенциалов и появлялся бы ток?\\

	
\end{document}

