\newpage
\centerline{\textbf{РЕФЕРАТ}}
\vspace{0.3cm}

\textit{Курсовая работа}, .

\vspace{0.1cm}

{ ЭЛЕКТРОСТАТИКА, ЭЛЕКТРИЧЕСКОЕ ПОЛЕ, МЕТОД КОНЕЧНЫХ ЭЛЕМЕНТОВ, АСИМПТОТИЧЕСКИЕ ГРАНИЧНЫЕ УСЛОВИЯ, УРАВНЕНИЕ ЛАПЛАСА, FENICS, PYTHON }

\vspace{0.1cm}

\textit{Объект исследования} ~--- двумерные электрические поля.

\vspace{0.1cm}

\textit{Методы исследования} ~--- использование свойств аналитических функций комплексного переменного.

\vspace{0.1cm}

\textit{Цель работы} ~--- расчёт потенциалов заряженных.

\vspace{0.1cm}

\textit{Результаты работы}:
\begin{itemize}
\item Выведена формула для комплексной напряжённости поля зарядов, лежащих на прямой;
\item Найдены выражения для комплексных потенциалов различных конфигураций двумерных электрических полей;
\item Построены графики изолиний для соответствующих полей.
\end{itemize}

\newpage

\centerline{\textbf{РЭФЕРАТ}}
\vspace{0.3cm}

\textit{Курсавая работа}, стр. 17, крынiц 2.

\vspace{0.1cm}

{ ЭЛЕКТРАСТАТЫКА, ЭЛЕКТРЫЧНАЕ ПОЛЕ, МЕТАД КАНЧАТКОВЫХ ЭЛЕМЕНТАУ, АСIМПТАТЫЧНЫЯ МЕЖАВЫЯ УМОВЫ, РАУНАННЕ ЛАПЛАСА, FENICS, PYTHON }

\vspace{0.1cm}

\textit{Аб'ект даследавання} ~--- двухмерныя электрычныя палі.

\vspace{0.1cm}

\textit{Метады даследвання} ~--- выкарыстанне ўласцівасцей аналітычных функцый комплекснага пераменнага.

\vspace{0.1cm}

\textit{Мэта работы} ~--- разлік асноўных канфігурацый двухмерных электрычных палёў з выкарыстаннем функцый комплекснага пераменнага.

\vspace{0.1cm}

\textit{Вынiкi работы}:
\begin{itemize}
\item Выведзена формула для комплекснай напружанасці поля зарадаў, якія ляжаць на прамой;
\item Знойдзены выразы для комплексных патэнцыялаў розных канфігурацый двухмерных электрычных палёў;
\item Пабудаваны графікі ізаліній для адпаведных палёў.
\end{itemize}

\newpage

%\chapter* {Abstract}
\centerline{\textbf{ABSTRACT}}
\vspace{0.3cm}

\textit {The course work}, pp. 17, sources 2.

\vspace {0.1cm}

{ELECTROSTATICS, ELECTRIC FIELD, FINITE ELEMENT METHOD, ASYMPTOTIC BOUDARY CONDITIONS, LAPLACE'S EQUATION, FENICS, PYTHON }

\vspace {0.1cm}

\textit {Subject of research} --- two-dimensional electric fields.

\vspace {0.1cm}

\textit {Methods} --- using properties of analytic functions of complex variables.

\vspace {0.1cm}

\textit {The purpose} --- calculation of basic configurations of two-dimensional electric field using a complex function.

\vspace {0.1cm}

\textit {The results}:
\begin {itemize}
\item The formula for the complex field intensity charges lying on the line;
\item Formulas for the complex potentials of different configurations of two-dimensional electric field;
\item Contour plots for the corresponding fields.
\end {itemize}
