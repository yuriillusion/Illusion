\newpage
\chapter{Расчёт потенциала заряженного кругового диска}
\section{Постановка задачи}
\begin{equation}
	\begin{cases}
		\Delta u = 0,\\
		u|_{\Gamma}=u_0,\\
		u\xrightarrow[r\rightarrow\infty]{}0.
	\end{cases}
\end{equation}

Заменим условие на бесконечности условием на достаточно большом радиусе $R$, так как заряженный диск на больших расстояниях можно считать точечным зарядом.

Воспользуемся тем, что решение не зависит от $\varphi$, симметрично относительно $\rho=0$ и $z=0$ и сведём трёхмерную задачу к двухмерной задаче на прямоугольнике $[0, R]\times[0,R]$.
\begin{equation}\label{disk_problem}
	\begin{cases}
		\Delta u = 0,\\
		\pdv{u}{\rho}\Bigr|_{\rho=0}=0,\\
		\pdv{u}{z}\Bigr|_{z=\pm 0}=I_{[0,r]}\cfrac{\sigma}{2\varepsilon_0},\\
		u|_{\Gamma_D}=\cfrac{\sigma r^2}{4 \varepsilon_0}\cfrac{1}{r}.
	\end{cases}
\end{equation}

\section{Вариационная формулировка задачи}
Оператор Лапласа в цилиндрических координатах:
\begin{equation}
	\Delta u = \cfrac{1}{\rho}\cfrac{\partial}{\partial \rho}\left(\rho\cfrac{\partial u}{\partial \rho}\right) + \cfrac{1}{\rho^2}\cfrac{\partial^2u}{\partial \varphi^2}+\cfrac{\partial^2u}{\partial z^2}.
\end{equation}

Домножим уравнение $\Delta u = 0$ на пробную функцию $v$ и проинтегрируем по области $\Omega = (0,R)\times(0,h)$ и рассмотрим левую часть:
\begin{equation}\label{laplace_cylindric}
	\int\limits_{\Omega_{X}} (\Delta_{(x,y,z)} u) v \dd{\Omega_{X}} = \int\limits_{\Omega} \left( \cfrac{\partial u}{\partial \rho}v + \cfrac{\partial^2u}{\partial \rho^2}v\rho + \cfrac{\partial^2u}{\partial z^2}v\rho \right) \dd{\Omega} = \int\limits_{\Omega} \left( \cfrac{\partial u}{\partial \rho}v + \Delta u v\rho \right) \dd{\Omega}.
\end{equation}
\begin{equation}\label{sub_laplace_cylindric}
	\int\limits_{\Omega} \left( \Delta u v\rho \right) \dd{\Omega} = \int\limits_{\partial \Omega}\rho v \pdv{u}{n}\dd{S}-\int\limits_{\Omega}\grad{u}\grad{v\rho}\dd{\Omega}=\int\limits_{\partial \Omega}\rho v \pdv{u}{n}\dd{S}-\int\limits_{\Omega}\grad{u}\grad{v}\rho\dd{\Omega}-\int\limits_{\Omega}\pdv{u}{\rho}v\dd{\Omega}.
\end{equation}
Подставляя (\ref{sub_laplace_cylindric}) в (\ref{laplace_cylindric}), получаем
\begin{equation}
	a(u,v)=\int\limits_{\partial \Omega}\rho v \pdv{u}{n}\dd{S}-\int\limits_{\Omega}\grad{u}\grad{v}\rho\dd{\Omega}.
\end{equation}
Таким образом, слабая формулировка задачи (\ref{disk_problem}):
\begin{equation}\label{disk_weak}
	\int\limits_{\Omega}\grad{u}\grad{v}\rho\dd{\Omega}-\int\limits_{\partial \Omega}\rho v \pdv{u}{n}\dd{S}=0.
\end{equation}
